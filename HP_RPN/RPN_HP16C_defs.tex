% ***************************************************
% Definition RPN code for HP-16C 
%
% (cl) 2013 by Arno Jacobs
%



% Main keys
\def \ONsi {\noindent ON }
\def \PRGMsi {\noindent \gshift P/R }

\def \tONsi      #1{\ONsi      & & & \rem {#1} \\ }
\def \tPRGMsi    #1{\PRGMsi    & & & \rem {#1} \\ }

% Basic functions without newline and indent codes
%

% line 1 left
% FOR A..F:  \def \NBRsic #1{ #1 }  ->  \NBRsic {A}

\def \SLsic   {\fshift SL  }
\def \SRsic   {\fshift SR  }
\def \RLsic   {\fshift RL  }
\def \RRsic   {\fshift RR  }
\def \RLnsic  {\fshift RLn }
\def \RRnsic  {\fshift RRn }

\def \LJsic   {\gshift LJ   }
\def \ASRsic  {\gshift ASR  }
\def \RLCsic  {\gshift RLC  }
\def \RRCsic  {\gshift RRC  }
\def \RLCnsic {\gshift RLCn }
\def \RRCnsic {\gshift RRCn }

% line 2 left
\def \GSBsic #1{GSB #1}
\def \GTOsic #1{GTO #1}
\def \HEXsic   {HEX }
\def \DECsic   {DEC }
\def \OCTsic   {OCT }
\def \BINsic   {BIN }

\def \Xswapindexisic {\fshift $x \lessgtr (i)$}
\def \XswapIsic      {\fshift $x \lessgtr I$}
\def \SHOWHEXsic {\fshift SHOW HEX }
\def \SHOWDECsic {\fshift SHOW DEC }
\def \SHOWOCTsic {\fshift SHOW OCT }
\def \SHOWBINsic {\fshift SHOW BIN }

\def \RTNsic    {\gshift RTN }
\def \LBLsic  #1{\gshift LBL #1}
\def \DSZsic    {\gshift DSZ }
\def \ISZsic    {\gshift ISZ }
\def \SqrtXsic  {\gshift $\sqrt{x}$ }
\def \OverXsic  {\gshift $1/x$ }


% line 3 left
\def \RSsic     {R/S}
\def \SSTsic    {SST}
\def \Rdownsic  {R$\downarrow$}
\def \XswapYsic {$x \lessgtr y$}
\def \BSPsic    {BSP}

\def \BiBsic       {\fshift (i)}
\def \Isic         {\fshift I}
\def \CLRPRGMsic   {\fshift CLEAR PRGM}
\def \CLRREGsic    {\fshift CLEAR REG}
\def \CLRPREFIXsic {\fshift CLEAR PREFIX}

\def \PRsic  {\gshift P/R}
\def \BSTsic {\gshift BST}
\def \Rupsic {\gshift R$\uparrow$}
\def \PSEsic {\gshift PSE}
\def \CLXsic {\gshift CL$X$}


% line 4 left
\def \STOsic  #1{STO #1}
\def \RCLsic  #1{RCL #1}

\def \WSIZEsic  {\fshift WSIZE }
\def \FLOATsic  {\fshift FLOAT }

\def \ALEFTsic  {\gshift $<$ }
\def \ARIGHTsic {\gshift $>$ }


% column 6 middle down
\def \ENTERsic    {ENTER}
\def \WINDOWsic   {\fshift WINDOW }
\def \LSTXsic     {\gshift LST$X$}


% line 1 right
\def \NBRsic #1{ #1 }

\def \MASKLsic   {\fshift MASKL }
\def \MASKRsic   {\fshift MASKR }
\def \RMDsic     {\fshift RMD  }
\def \XORsic     {\fshift XOR  }

\def \HASHBsic   {\gshift $\#$B }
\def \ABSsic     {\gshift ABS }
\def \DBLRsic    {\gshift DBLR  }
\def \DBLdivsic  {\gshift DBL$\div$ }

% line 2 right
\def \SBsic      {\fshift SB }
\def \CBsic      {\fshift CB }
\def \Bitsic     {\fshift B?  }
\def \ANDsic     {\fshift AND  }

\def \SFsic   #1{\gshift SF {#1} }
\def \CFsic   #1{\gshift CF {#1} }
\def \Flagsic #1{\gshift F? {#1} }
\def \DBLxsic   {\gshift DBL$\times$ }

% line 3 right
\def \SConeSsic   {\fshift Set Compl $1'S$ }
\def \SCtwoSsic   {\fshift Set Compl $2'S$ }
\def \SCUNSGNsic  {\fshift Set Compl UNSGN }
\def \NOTsic      {\fshift NOT} 

\def \XsmlreqYsic     {\gshift X$\leqslant$Y }
\def \XsmlrZEROsic    {\gshift X$<$0 }
\def \XgreaterYsic    {\gshift X$>$Y }
\def \XgreaterZEROsic {\gshift X$>$0 }


% line 4 right
\def \Pointsic    {$.$ }
\def \CHSsic      {CHS }

\def \MEMsic      {\fshift MEM }
\def \STATUSsic   {\fshift STATUS }
\def \EEXsic      {\fshift EEX }
\def \ORsic       {\fshift OR }

\def \XnoteqYsic     {\gshift X$\notequal$Y }
\def \XnoteqZEROsic  {\gshift X$\notequal$0 }
\def \XeqYsic        {\gshift X$=$Y }
\def \XeqZEROsic     {\gshift X$=$0 }


% last column operators
\def \OPdividesic    { $\div$ }
\def \OPmultiplysic  { $\times$ }
\def \OPminussic     { - }
\def \OPplussic      { + }




% Basic functions including layout codes and the tabular versions (starting with 't')

% Numbers 0..9, a..F only
\def \tNBRsi  #1 #2{\cpp \indent \NBRsic {#1}  & \lnbr{linenumber} & #1        & \rem {#2} \\}

% line 1 left
\def \tSLsi   #1{\cpp \indent \SLsic   & \lnbr{linenumber} & 42 11       & \rem {#1} \\}
\def \tSRsi   #1{\cpp \indent \SRsic   & \lnbr{linenumber} & 42 12       & \rem {#1} \\}
\def \tRLsi   #1{\cpp \indent \RLsic   & \lnbr{linenumber} & 42 13       & \rem {#1} \\}
\def \tRRsi   #1{\cpp \indent \RRsic   & \lnbr{linenumber} & 42 14       & \rem {#1} \\}
\def \tRLnsi  #1{\cpp \indent \RLnsic  & \lnbr{linenumber} & 42 15       & \rem {#1} \\}
\def \tRRnsi  #1{\cpp \indent \RRnsic  & \lnbr{linenumber} & 42 16       & \rem {#1} \\}

\def \tLJsi   #1{\cpp \indent \LJsic   & \lnbr{linenumber} & 43 11       & \rem {#1} \\}
\def \tASRsi  #1{\cpp \indent \ASRsic  & \lnbr{linenumber} & 43 12       & \rem {#1} \\}
\def \tRLCsi  #1{\cpp \indent \RLCsic  & \lnbr{linenumber} & 43 13       & \rem {#1} \\}
\def \tRRCsi  #1{\cpp \indent \RRCsic  & \lnbr{linenumber} & 43 14       & \rem {#1} \\}
\def \tRLCnsi #1{\cpp \indent \RLCnsic & \lnbr{linenumber} & 43 15       & \rem {#1} \\}
\def \tRRCnsi #1{\cpp \indent \RRCnsic & \lnbr{linenumber} & 43 16       & \rem {#1} \\}


% line 2 left
\def \tGSBsi #1 #2{\cpp \indent \GSBsic {#1} & \lnbr{linenumber} & 21  #1   & \rem {#2} \\}
\def \tGTOsi #1 #2{\cpp \indent \GTOsic {#1} & \lnbr{linenumber} & 22  #1   & \rem {#2} \\}
\def \tHEXsi    #1{\cpp \indent \HEXsic      & \lnbr{linenumber} & 23       & \rem {#1} \\}
\def \tDECsi    #1{\cpp \indent \DECsic      & \lnbr{linenumber} & 24       & \rem {#1} \\}
\def \tOCTsi    #1{\cpp \indent \OCTsic      & \lnbr{linenumber} & 25       & \rem {#1} \\}
\def \tBINsi    #1{\cpp \indent \BINsic      & \lnbr{linenumber} & 26       & \rem {#1} \\}

\def \tXswapindexisi #1{\cpp \indent \Xswapindexisic  & \lnbr{linenumber} & 42 21       & \rem {#1} \\}
\def \tXswapIsi      #1{\cpp \indent \XswapIsic       & \lnbr{linenumber} & 42 22       & \rem {#1} \\}
\def \tSHOWHEXsi     #1{\cpp \indent \SHOWHEXsic      & \lnbr{linenumber} & 42 23       & \rem {#1} \\}
\def \tSHOWDECsi     #1{\cpp \indent \SHOWDECsic      & \lnbr{linenumber} & 42 24       & \rem {#1} \\}
\def \tSHOWOCTsi     #1{\cpp \indent \SHOWOCTsic      & \lnbr{linenumber} & 42 25       & \rem {#1} \\}
\def \tSHOWBINsi     #1{\cpp \indent \SHOWBINsic      & \lnbr{linenumber} & 42 26       & \rem {#1} \\}

\def \tRTNsi     #1{\cpp \indent \RTNsic      & \lnbr{linenumber} & 43 21       & \rem {#1} \\}
\def \tLBLsi  #1 #2{\cpp \indent \LBLsic {#1} & \lnbr{linenumber} & 43,22, #1   & \rem {#2} \\}
\def \tDSZsi     #1{\cpp \indent \DSZsic      & \lnbr{linenumber} & 43 23       & \rem {#1} \\}
\def \tISZsi     #1{\cpp \indent \ISZsic      & \lnbr{linenumber} & 43 24       & \rem {#1} \\}
\def \tSqrtXsi   #1{\cpp \indent \SqrtXsic    & \lnbr{linenumber} & 43 25       & \rem {#1} \\}
\def \tOverXsi   #1{\cpp \indent \OverXsic    & \lnbr{linenumber} & 43 26       & \rem {#1} \\}


% line 3 left
\def \tRSsi       #1{\cpp \indent  \RSsic       & \lnbr{linenumber} & 31       & \rem {#1} \\}
\def \tSSTsi      #1{\noindent     \SSTsic      & {}                & {}       & \rem {#1} \\}
\def \tRdownsi    #1{\cpp \indent  \Rdownsic    & \lnbr{linenumber} & 33       & \rem {#1} \\}
\def \tXswapYsi   #1{\cpp \indent  \XswapYsic   & \lnbr{linenumber} & 34       & \rem {#1} \\}
\def \tBSPsi      #1{\cpp \indent  \BSPsic      & {}                & {}       & \rem {#1} \\}

\def \tBiBsi        #1{\cpp \indent \BiBsic       & \lnbr{linenumber} & 42 31    & \rem {#1} \\}
\def \tIsi          #1{\cpp \indent \Isic         & \lnbr{linenumber} & 42 32    & \rem {#1} \\}
\def \tCLRPRGMsi    #1{\noindent    \CLRPRGMsic   & {}                & {}       & \rem {#1} \\}
\def \tCLRREGsi     #1{\cpp \indent \CLRREGsic    & \lnbr{linenumber} & 42 34    & \rem {#1} \\}
\def \tCLRPREFIXsi  #1{\cpp \indent \CLRPREFIXsic & \lnbr{linenumber} & 42 35    & \rem {#1} \\}

\def \tPRsi         #1{\cpp \indent \PRsic        & {}                & {}       & \rem {#1} \\}  
\def \tBSTsi        #1{\cpp \indent \BSTsic       & \lnbr{linenumber} & 43 32    & \rem {#1} \\}
\def \tRupsi        #1{\cpp \indent \Rupsic       & \lnbr{linenumber} & 43 33    & \rem {#1} \\}
\def \tPSEsi        #1{\cpp \indent \PSEsic       & \lnbr{linenumber} & 43 34    & \rem {#1} \\}
\def \tCLXsi        #1{\cpp \indent \CLXsic       & \lnbr{linenumber} & 43 35    & \rem {#1} \\}


% line 4 left
\def \tSTOsi  #1 #2{\cpp \indent \STOsic {#1}    & \lnbr{linenumber} & 44  #1    & \rem {#2} \\}
\def \tRCLsi  #1 #2{\cpp \indent \RCLsic {#1}    & \lnbr{linenumber} & 45  #1    & \rem {#2} \\}

\def \tWSIZEsi      #1{\cpp \indent \WSIZEsic    & \lnbr{linenumber} & 42 44     & \rem {#1} \\}
\def \tFLOATsi      #1{\cpp \indent \FLOATsic    & \lnbr{linenumber} & 42 45     & \rem {#1} \\}

\def \tALEFTsi      #1{\cpp \indent \ALEFTsic    & \lnbr{linenumber} & 43 44     & \rem {#1} \\}
\def \tARIGHTsi     #1{\cpp \indent \ARIGHTsic   & \lnbr{linenumber} & 43 45     & \rem {#1} \\}


% column 6 middle down
\def \tENTERsi     #1{\cpp \indent \ENTERsic     & \lnbr{linenumber} & 36         & \rem {#1} \\}
\def \tWINDOWsi    #1{\cpp \indent \WINDOWsic    & \lnbr{linenumber} & 42 36      & \rem {#1} \\}
\def \tLSTXsi      #1{\cpp \indent \LSTXsic      & \lnbr{linenumber} & 43 36      & \rem {#1} \\}


% line 1 right
\def \tMASKLsi   #1{\cpp \indent \MASKLsic      & \lnbr{linenumber} & 42  7     & \rem {#1} \\}
\def \tMASKRsi   #1{\cpp \indent \MASKRsic      & \lnbr{linenumber} & 42  8     & \rem {#1} \\}
\def \tRMDsi     #1{\cpp \indent \RMDsic        & \lnbr{linenumber} & 42  9     & \rem {#1} \\}
\def \tXORsi     #1{\cpp \indent \XORsic        & \lnbr{linenumber} & 42 10     & \rem {#1} \\}

\def \tHASHBsi   #1{\cpp \indent \HASHBsic      & \lnbr{linenumber} & 43  7     & \rem {#1} \\}
\def \tABSsi     #1{\cpp \indent \ABSsic        & \lnbr{linenumber} & 43  8     & \rem {#1} \\}
\def \tDBLRsi    #1{\cpp \indent \DBLRsic       & \lnbr{linenumber} & 43  9     & \rem {#1} \\}
\def \tDBLdivsi  #1{\cpp \indent \DBLdivsic     & \lnbr{linenumber} & 43 10     & \rem {#1} \\}


% line 2 right
\def \tSBsi   #1{\cpp \indent \SBsic      & \lnbr{linenumber} & 42  4     & \rem {#1} \\}
\def \tCBsi   #1{\cpp \indent \CBsic      & \lnbr{linenumber} & 42  5     & \rem {#1} \\}
\def \tBitsi  #1{\cpp \indent \Bitsic     & \lnbr{linenumber} & 42  6     & \rem {#1} \\}
\def \tANDsi  #1{\cpp \indent \ANDsic     & \lnbr{linenumber} & 42 20     & \rem {#1} \\}

\def \tSFsi    #1 #2{\cpp \indent \SFsic {#1}   & \lnbr{linenumber} & 43, 4, #1   & \rem {#2} \\}
\def \tCFsi    #1 #2{\cpp \indent \CFsic {#1}   & \lnbr{linenumber} & 43, 5, #1   & \rem {#2} \\}
\def \tFlagsi  #1 #2{\cpp \indent \Flagsic {#1} & \lnbr{linenumber} & 43, 6, #1   & \rem {#2} \\}
\def \tDBLxsi     #1{\cpp \indent \DBLxsic      & \lnbr{linenumber} & 43 20       & \rem {#1} \\}


% line 3 right
\def \tSConeSsi   #1{\cpp \indent \SConeSsic      & \lnbr{linenumber} & 42  1     & \rem {#1} \\}
\def \tSCtwoSsi   #1{\cpp \indent \SCtwoSsic      & \lnbr{linenumber} & 42  2     & \rem {#1} \\}
\def \tSCUNSGNsi  #1{\cpp \indent \SCUNSGNsic     & \lnbr{linenumber} & 42  3     & \rem {#1} \\}
\def \tNOTsi      #1{\cpp \indent \NOTsic         & \lnbr{linenumber} & 42 30     & \rem {#1} \\} 

\def \tXsmlreqYsi     #1{\cpp \indent \XsmlreqYsic     & \lnbr{linenumber} & 43  1     & \rem {#1} \\}
\def \tXsmlrZEROsi    #1{\cpp \indent \XsmlrZEROsic    & \lnbr{linenumber} & 43  2     & \rem {#1} \\}
\def \tXgreaterYsi    #1{\cpp \indent \XgreaterYsic    & \lnbr{linenumber} & 43  3     & \rem {#1} \\}
\def \tXgreaterZEROsi #1{\cpp \indent \XgreaterZEROsic & \lnbr{linenumber} & 43 30     & \rem {#1} \\}

% line 4 right
\def \tPointsi        #1{\indent \Pointsic             & {}                & {}        & \rem {#1} \\}
\def \tCHSsi          #1{\cpp \indent \CHSsic          & \lnbr{linenumber} & 49        & \rem {#1} \\}


\def \tMEMsi      #1{\indent \MEMsic            & {}                & {}        & \rem {#1} \\}
\def \tSTATUSsi   #1{\indent \STATUSsic         & {}                & {}        & \rem {#1} \\}
\def \tEEXsi      #1{\cpp \indent \EEXsic       & \lnbr{linenumber} & 42 49     & \rem {#1} \\}
\def \tORsi       #1{\cpp \indent \ORsic        & \lnbr{linenumber} & 42 40     & \rem {#1} \\}

\def \tXnoteqYsi     #1{\cpp \indent \XnoteqYsic      & \lnbr{linenumber} & 43  0     & \rem {#1} \\}
\def \tXnoteqZEROsi  #1{\cpp \indent \XnoteqZEROsic   & \lnbr{linenumber} & 43 48     & \rem {#1} \\}
\def \tXeqYsi        #1{\cpp \indent \XeqYsic         & \lnbr{linenumber} & 43 49     & \rem {#1} \\}
\def \tXeqZEROsi     #1{\cpp \indent \XeqZEROsic      & \lnbr{linenumber} & 43 40     & \rem {#1} \\}



% last column - operators
\def \tOPdividesi     #1{\cpp \indent \OPdividesic    & \lnbr{linenumber} & 10     & \rem {#1} \\}
\def \tOPmultiplysi   #1{\cpp \indent \OPmultiplysic  & \lnbr{linenumber} & 20     & \rem {#1} \\}
\def \tOPminussi      #1{\cpp \indent \OPminussic     & \lnbr{linenumber} & 30     & \rem {#1} \\}
\def \tOPplussi       #1{\cpp \indent \OPplussic      & \lnbr{linenumber} & 40     & \rem {#1} \\}



% End of definitions
